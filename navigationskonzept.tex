\section{Navigationskonzept - MV}

\textsc{Matthias Vongerichten}

\subsection{Techniche Umsetzung einheitlicher Navigation}

F�r jede Seite existiert eine PHP-Datei (Content) mit dem jeweiligen Inhalt, die au�erdem den Header und den Footer einbindet. Header und Footer werden somit f�r alle Seiten vereinheitlicht dargestellt und k�nnen zentral bearbeitet werden.

\subsection{Seitenhierarchie}

Die Website besteht aus max. 2 Ebenen. Lediglich die Reisedetails geh�ren der 2.ten Ebene an. Auf diese gelangt man �ber die Reiseziele.

\subsection{Hauptmen�}

Das Hauptmen� ist im Header angesiedelt und beinhaltet nur Seiten der Ebene 1.

\subsection{Breadcrumb Navigation}

Zur besseren Orientierung wird �ber dem Titel einer Seite eine "`Brotkr�mel"'-Navigation angezeigt. Diese ist im Content angesiedelt.

\subsection{Related Pages}

Je Reiseziel werden in einem gesonderten Abschnitt Hyperlinks zu relevanten Internetseiten angezeigt. Diese beziehen sich immer auf die ausgew�hlte Reise und werden daher im Content realisiert.

\subsection{Suchmaschinenoptimierung - SEO}

Um die Popularit�t der Site zu erh�hen werden heutzutage kaum noch META-tags verwendet. Sinnvoll sind diese nur noch als Vorgabe des Beschreibungstextes in den Suchergebnissen  ("`description"'-tag) und bei einer gew�nschten Einschr�nkung der Suchmaschinen-Robots ("`robots"'-tag). Da eine Einschr�nkung der Suchmaschinen nicht erw�nscht ist, sollte lediglich das "`description"'-tag verwendet werden.

Um die Relevanz der Site zu erh�hen, sollte(n)

\begin{itemize}
	\item der Inhalt regelm��ig aktualisiert werden
	\item die Reisen ausf�hrlich beschrieben sein
	\item daf�r gesorgt werden, dass die Website auf anderen, qualitativ hochwertigen Seiten verlinkt wurde (Backlinks)
	\item es die M�glichkeit geben Seiten zu teilen (Facebook, Google+ usw.)
\end{itemize}



