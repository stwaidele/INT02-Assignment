\section{Fazit}

\textsc{Matthias Vongerichten}

Das redaktionelle Konzept beschreibt die vorgesehenen Inhalte je Seite in Kurzform. Au�erdem wird hierbei auch eine Richtlinie zur Aktualisierungsh�ufigkeit je Seite vorgegeben (unterst�tzt SEO). Somit wird erreicht, dass der redaktionelle Verantwortliche eine kompakte Anleitung zur Pflege der Website erh�lt.

Beim Navigationskonzept wurden die wichtigsten Navigationselemente beachtet. Hervorzuheben sind hier die "`Breadcrumb"'-Navigation und die Implementierung von mod\_rewrite.

\textsc{Stefan Waidele}

...

\subsection{Evt. Gesamtfazit}

Die Website erf�llt die gestellten Anforderungen vollst�ndig. So bescheinigt uns woorank.com (SEO-Analyse-Tool), dass z.B. nicht-funktionalen Anforderungen wie die Performance der Site sehr gut erf�llt werden. %screenshot 

Ein n�chster m�glicher Schritt ist die Verwendung eines Content-Management-Systems (CMS). Dies erm�glicht das einfache Bearbeiten der Seiteninhalte mithilfe eines WYSIWYG-Editors (keine Code-Kenntnisse notwendig) und das Aufrechterhalten eines konsistenten Designs. Mit dem CMS ist es ebenfalls m�glich einen Weblog zu betreiben welcher mit Facebook und anderen Social-Media-Sites integriert werden kann. Der Administrationsaufwand wird somit stark reduziert und trotzdem f�llt es erstaunlich leicht die Funktionen der Website durch vorgefertigte Module zu erweitern.
