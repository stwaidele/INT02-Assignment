\section{Detailbetrachtung: Dropdown--Men�s mit CSS}
\label{sec:cssdropdown}
\textsc{Stefan Waidele}

Wie im Designkonzept beschrieben, wird das Dropdown--Men� in der Navigationsleiste ohne JavaScript\footnote{Vgl. Kapitel \ref{sec:avoidjs}, Seite \pageref{sec:avoidjs}} implementiert, auch wenn hierbei geringe Abstriche bei der Browserkompatibilit�t zu machen sind\footnote{Vgl. Kapitel \ref{sec:compat}, Seite \pageref{sec:compat}}. Die dahinter stehende Technik soll hier in Grundz�gen gezeigt werden\footnote{Vgl. \cite{ericmeyer}}. Der �bersichtlichkeit halber werden nur die relevanten Tags bzw. Attribute in den Codebeispielen aufgef�hrt.

Die Men�leiste wird, semantisch korrekt, als Aufz�hlungsliste im HTML--Quelltext ausgezeichnet, und dann mit den CSS--Eigenschaften entsprechend formatiert. So entsteht die Hauptebene der Navigation mit den Seiten der ersten Hierarchiestufe.

\begin{figure}[h]
\begin{minted}[bgcolor=bg]{html}
<div id="nav">
   <ul>
      <li><a href="/">Start</a></li>
      <li><a href="/reiseziele">Reiseziele</a></li>
         <ul>
            <li>In Deutschland</li>
            <li><a href="/reiseziele/berlin">Berlin</a></li>
            <li>oder europaweit</li>
            <li><a href="/reiseziele/amsterdam">Amsterdam</a></li>
         </ul>
      <li><a href="/busse">Die Busse</a></li>
      <li><a href="/wirueberuns">�ber uns</a></li>
      <li><a href="/kontakt">Kontakt</a></li>
   </ul>
</div>
\end{minted}
\caption{Quellcode: Navigationsmen�, Ebene 1 und 2 (HTML)}
\label{abb:navmenul1}
\end{figure} 

Die zweite Hierarchiestufe wird, wie in Abbildung~\ref{abb:navmenul1} gezeigt, in einer geschachtelten Liste ausgezeichnet.  

Mit dem CSS-Selektor \code{\#nav li ul}\footnote{Alle nicht--nummerierten Listen, die in einem \code{li}-Tag innerhalb einem \code{DIV} mit der ID \code{nav} enthalten sind} sowie der CSS-Eigenschaft \code{display: none;} wird diese zun�chst ausgeblendet.

�ber den Selektor \code{\#nav li:hover > ul}\footnote{Die erste nicht--nummerierte Liste, die sich in dem \code{li}-Tag unterhalb des Mauszeigers innerhalb einem \code{DIV} mit der ID \code{nav} enthalten ist} wird das Men� genau dann mit \code{display: block;} angezeigt, wenn sich die Maus �ber dem Men�titel befindet.

Die zweite Hierarchiestufe wird nur dann eingeblendet, wenn der Mauszeiger ein entsprechendes Element der obersten Ebene ber�hrt. 

\subsection{Browserkompatibilit�t}

Die hier gezeigte L�sung funktioniert in allen Browsern. Hierbei sind folgende F�lle zu unterscheiden:

\begin{itemize}
	\item \textbf{Browser ohne CSS--Unterst�tzung:}\\Browser, die mit CSS gar nicht umgehen k�nnen, stellen den HTML--Code komplett, d.h. der Nutzer sieht das komplette Men�. Dieses wird dann aber als normal formatierte, geschachtelte Liste untereinander mit Spiegelpunkten angezeigt. Die gesamte Seite sieht nicht so aus, wie dies vom Betreiber gew�nscht ist, aber es werden alle Informationen und die vollst�ndige Navigation angezeigt.
	\item \textbf{Alte Browser mit schlechter CSS--Unterst�tzung:}\\Alle Browser, die CSS unterst�tzen k�nnen mit der Eigenschaft \code{display: none;} umgehen. In den Anfangszeiten der CSS--Unterst�tzung wurde jedoch die \code{:hover} Pseudoklasse nicht f�r alle Elemente vorgesehen. Somit wird in diesen Browsern nur die oberste Men�ebene angezeigt\footnote{Vgl. \cite{ericmeyer}}. F�r Anwender mit solch historischen Browsern sollten die tieferen Hierarchieebenen �ber die Seitenleiste oder anderweitig nochmals verlinkt sein, damit auch hier die komplette Website angezeigt werden kann.
	\item \textbf{Moderne Browser:}\\Browser mit vollst�ndiger CSS--Unterst�tzung zeigen die Men�s (und die gesamte Website) so an, wie dies vom Betreiber und Designer vorgesehen ist. Da die hier genutzten Techniken bereits schon seit �ber 10 Jahren existieren, sollten somit die meisten Nutzer inzwischen keine Probleme mit der Darstellung haben.
\end{itemize}	
\clearpage