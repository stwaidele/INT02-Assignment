\section{Annahmen}

\textsc{Stefan Waidele}

Die Unternehmenswebsite f�gt sich in die gro�e Bandbreite der Werbema�nahmen eines Unternehmens ein.
Daher muss diese sich auch nach den Grunds�tzen der Werbung und des Marketings richten.

F�r diese Arbeit gehen wir von folgendenden Vorgaben \footnote{\cite{kloss}, S. 183ff} aus:

\begin{itemize}
	\item \textbf{Zielgruppe:}\\
	Das Unternehmen betrachtet Alleinreisende und Paare im Alter zwischen 40 und 65 Jahren als die Hauptzielgruppe f�r die angebotenen St�dtereisen. Au�erdem sollen Vereine und Reisegruppen gemischten Alters angesprochen werden. 
	Altersunabh�ngig ist die Zielgruppe haupts�chlich den Sinus-Charakteristiken "`B�rgerliche Mitte"', "`Traditionelle"' oder "`Konservativ--etablierte"' zuzuordnen\footnote{\cite{sinus}}.
	
	\item \textbf{Phase der Buchungsentscheidung:}\\
	Die Website soll dem Kunden zu jedem Zeitpunkt des Entscheidungsvorgangs\footnote{vgl. \cite{freyer}, S105} etwas bieten k�nnen. Es ist jedoch besonders auf die Informations-- und Handlungsphase zu achten, da hier der Einfluss des Unternehmens am direktesten ist.
\end{itemize}



