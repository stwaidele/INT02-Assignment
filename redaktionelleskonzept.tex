\section{Redaktionelles Konzept}

\textsc{Matthias Vongerichten bzw. Stefan Waidele}

\subsection{Startseite - MV}

\textsc{Titel: Willkommen}\newline
\textsc{Updatezyklus: halbj�hrlich}

Die Startseite hei�t den Besucher auf der Website willkommen. Hierf�r wird ein mehrzeiliger Flie�text angezeigt. In diesem wird die Vision und die Leidenschaft der Unternehmung beschrieben. Zus�tzlich wird im Flie�text ein Foto einer bereits durchgef�hrten Reise angezeigt.

\subsection{Unternehmen - MV}

\textsc{Titel: Wir stellen uns vor}\newline
\textsc{Updatezyklus: bei Notwendigkeit}

Diese Seite wird auf 3 Abschnitte unterteilt. 

\subsubsection{Abschnitt: Was uns auszeichnet}

Gefordert ist hier, dass herauskristallisiert wird was einem von der Konkurrenz unterscheidet bzw. abhebt. Auf was wird besonderen Wert gelegt? F�r was ist man bekannt?

\subsubsection{Abschnitt: Das Team ...}

Ein Gruppenbild des Teams gen�gt hier.

\subsubsection{Abschnitt: ... und unsere Geschichte}

Historische Entwicklung des Unternehmens. Was waren die "`Meilensteine"'? Wie kam es zur Gr�ndung? Auf was hatte man sich am Anfang spezialisiert und welchen Umfang bietet man heute? 

\subsection{Reiseziele - MV}

\textsc{Titel: Wohin soll es gehen?}\newline
\textsc{Updatezyklus: bei Notwendigkeit}

Hier findet eine erste Gliederung nach L�ndern statt. Je Land werden die angebotenen Reisen als kleiner Ausschnitt dargestellt. Der Ausschnitt zeigt ein aussagekr�ftiges Bild, eine kurze Beschreibung sowie Hyperlink zu den Reisedetails.

\subsection{Reisedetails - MV}

\textsc{Titel: [Titel der Reise]}\newline
\textsc{Updatezyklus: bei Notwendigkeit}

Hier erh�lt man genauere Informationen �ber eine Reise. 
Auf der linken Seite sind mindestens 3 kleine Bilder vorhanden, welche vom Beschreibungstext umflossen werden. Der Text besteht aus einer kurzen Einf�hrung am Anfang, geht in die Tagesplanung �ber und endet mit einer Auflistung folgender Elemente \footnote{\cite{freyer}, S103}, die die Entscheidungsphase ma�geblich unterst�tzen:

\begin{description}
		\item[Reisezeitpunkt]
		\item[Reisedauer]
		\item[Reisegebiet]
		\item[Reisegestaltung] Hier handelt es sich immer um Pauschalangebote
		\item[Kosten der Reise] Angabe pro Person
		\item[Reisekomfort] Angabe der verwendeten Busklasse
		\item[Unterkunftsart]
		\item[Nebenleistungen und - ausgaben]
\end{description}

\subsubsection{Abschnitt: Termine}

Hier werden mindestens die Termine des aktuellen Jahres aufgelistet.

\subsubsection{Abschnitt: Angebot}

Eine Auflistung der Leistungen, wie z.B. die �bernachtungsm�glichkeit, Verpflegung, Besuche von Einrichtungen usw., wird auf der rechten Seite erg�nzt durch den Preis der Reise und ein dynamischer Hyperlink, welcher auf die Buchungsseite weiterleitet mit Weitergabe der Angebotsnummer. 


\subsection{Informationen zu den Bussen}

...

\subsection{Buchungsm�glichkeiten}

\subsubsection{Telefon}
\subsubsection{Schriftlich: Brief, Fax, E-Mail}
Evt. Zitat aus meinem ANS03-Assignment: "`E-Mail = Brief"'

\subsubsection{HTML--Formular}

...

\subsubsection{Internet Booking Engine -- IBE}

...

\subsection{Social--Media Einbindung}

Kein eigenes Social--Media--Konzept, da nur Website gefordert

Twitter-Feed auf Homepage oder als eigene Seite (gefiltert nach Hashtag) --- Pro und Contra

Kommentarfunktion bei einzelnen Reisen

Einbindung von Bewertungsportalen (Customer--Alliance Demo-Siegel?)

\subsection{Noch Konzept...}

...
