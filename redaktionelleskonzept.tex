\section{Redaktionelles Konzept}

\textsc{Matthias Vongerichten}

\subsection{Start}

\textsc{Updatezyklus: j�hrlich}

Die Startseite hei�t den Besucher auf der Website willkommen. Hierf�r wird ein mehrzeiliger Flie�text angezeigt. In diesem wird die Vision und die Leidenschaft der Unternehmung beschrieben. Zus�tzlich wird im Flie�text ein Foto einer k�rzlich durchgef�hrten Reise angezeigt.

\subsection{�ber uns}

\textsc{Updatezyklus: bei Notwendigkeit}

Diese Seite wird auf 3 Abschnitte unterteilt: 

\begin{itemize}
\item \textbf{Abschnitt: Was uns auszeichnet}\\
Gefordert ist hier, dass herauskristallisiert wird was einem von der Konkurrenz unterscheidet bzw. abhebt. Auf was wird besonderen Wert gelegt? F�r was ist man bekannt?
\item \textbf{Abschnitt: Das Team ...}\\
Hier wird das Team in Form eines Gruppenbilds vorgestellt.
\item \textbf{Abschnitt: ... und unsere Geschichte}\\
Historische Entwicklung des Unternehmens. Was waren die "`Meilensteine"'? Wie kam es zur Gr�ndung? Auf was hatte man sich am Anfang spezialisiert und welchen Umfang bietet man heute? 
\end{itemize}

\subsection{Reiseziele}

\textsc{Updatezyklus: bei Notwendigkeit}

Hier werden die momentan besonders hervorzuhebenden Reisen, sortiert nach Zielen in Deutschland und Europa aufgelistet.
Die angebotenen St�dtereisen werden als kleiner Ausschnitt dargestellt. Der Ausschnitt zeigt ein aussagekr�ftiges Bild, eine kurze Beschreibung sowie einen Hyperlink zu den Reisedetails.

Auf Unterseiten werden nach L�ndern gegliedert alle angebotenen Reisen angezeigt. 

\subsection{Reisedetails}

\textsc{Updatezyklus: bei Notwendigkeit}

Hier erh�lt man genauere Informationen �ber eine St�dtereise. 
Auf der linken Seite sind mindestens 2 kleine Bilder vorhanden, welche vom Beschreibungstext umflossen werden. Der Text besteht aus einer kurzen Einf�hrung am Anfang und geht in die Tagesplanung �ber. In der Sidebar auf der rechten Seite erfolgt eine detaillierte Auflistung der Reisedaten. Folgende Elemente \footnote{\cite{freyer}, S103}, die die Entscheidungsphase ma�geblich unterst�tzen, sollen auf der Seite Erw�hnung finden: 

Reisezeitpunkt, Reisedauer, Reisegebiet, Reisegestaltung (Hier handelt es sich immer um Pauschalangebote), Kosten der Reise pro Person, Reisekomfort (Angabe der verwendeten Busklasse, Unterkunftsart, Nebenleistungen und -ausgaben.

Zus�tzlich wird ein Hyperlink zur Buchungsanfrage angezeigt. Der Link verweist hauf das �bliche Kontaktformular und �bergibt hierbei noch die ausgew�hlte St�dtereise als Parameter.

\clearpage
\subsection{Die Busse}

\textsc{Updatezyklus: bei Notwendigkeit}

Hier kann sich der potenzielle Kunde einen �berblick �ber den Fuhrpark des Unternehmens verschaffen. Ein Bild je Bus sowie dessen besonderen Merkmale stellen ein Minimum dar.

\subsection{Kontakt}

\textsc{Updatezyklus: bei Notwendigkeit}

Die Seite erm�glicht es dem Anwender auf verschiedene Weise mit dem Unternehmen in Kontakt zu treten. Angeboten wird:

Ein Kontaktformular, die E--Mail Adresse, Telefonnummer, Faxnummer, Links zu Social Media Plattformen

\subsection{Impressum}

\textsc{Updatezyklus: bei Notwendigkeit}

Nach Paragraf 5 des Telemediengesetz\footnote{\cite{justlaw}} sind folgende Angaben notwendig:

Name und Anschrift, E--Mail Adresse, Telefon, ggf. Register und Registernummer, 
ggf. Umsatzsteuer-- oder Wirtschaftssteueridentifikationsnummer

\subsection{AGB}

\textsc{Updatezyklus: bei Notwendigkeit}

Allgemeine Gesch�ftsbedingungen erweitern gesetzliche Regelungen nach eigenem Ermessen (innerhalb "`der wesentlichen Grundgedanken"') und sind daher freiwillig anzugeben. Im Falle dieses Busunternehmens k�nnte grunds�tzlich auf eine AGB verzichtet werden, jedoch bietet diese Seite auch die M�glichkeit sonstigen Informationspflichten bei Internetangeboten nachzugehen\footnote{\cite{tschwenke}}:

Widerrufsbelehrungspflichten gegen�ber Verbrauchern, Hinweise wie der Vertrag zustande kommt, 
Informationspflichten betreffend Preisangaben, Datenschutzhinweise, Hinweise nach der 
Dienstleistungs-Informationspflichten-Verordnung

\subsection{Social Media}

Eine Pr�senz in den sozialen Netzwerken geh�rt heutzutage in jedes erfolgreiche Website-Konzept. Neben der einfachen M�glichkeit von Kunden offiziell "`empfohlen"' zu werden bietet sich die M�glichkeit, als Alternative zu einem Weblog, potenzielle Kunden mit Neuigkeiten zu versorgen und mit Ihnen schnell in Kontakt zu treten. Aufgrund seiner aktuell noch dominierenden Stellung wird ein Account bei Facebook empfohlen - bei steigender Nutzeranzahl sp�ter auch ein Google+-Account, welcher dann mit Facebook integriert werden kann.

Im Header der Site kann die Facebook-Page besucht und "`geliked"' sowie bei Google+ - da dort noch kein eigener Account existiert - die Domain "`+1"'-ed werden. 