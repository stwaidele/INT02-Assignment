\section{Einleitung}
\subsection{Ziel der Arbeit}

Ziel dieser Arbeit ist die Erstellung einer Website f�r ein fiktives Busunternehmen ohne Zuhilfenahme von speziellen WYSIWYG Website-Editoren oder CMS--Systemen.

F�r die Website sind jeweils ein redaktionelles, ein Gestaltungs-- und ein Navigationskonzept zu erstellen. Von jedem Teammitglied ist eine Einzelseite des Internetauftritts zu erstellen.

Das Ziel dieser Arbeit ergibt sich direkt aus der Aufgabenstellung im Rahmen des AKAD--Studienmoduls "`INT02 -- Einf�hrung in die Internetprogrammierung"'

\subsection{Vorgehensweise}

\begin{itemize}
	\item Im Kapitel "`Annahmen"' werden die Grundlagen f�r die zu erarbeitenden Konzepte kurz beschrieben.
	\item In den folgenden Kapiteln wird die Website bez�glich den Inhalten, der Navigation und des Designs geplant.
	\item Jeder der beiden Autoren realisiert eine konkrete Seite und beschreibt die hierzu eingesetzten Methoden.
\end{itemize}


\subsection{Abgrenzung}

Lediglich zwei Seiten des Internetauftritts werden im Rahmen dieser Arbeit im Detail betrachtet.
Die �brigen werden lediglich in den zu erarbeitenden Konzepten besprochen, ohne jedoch tats�chlich erstellt zu werden.

Ebenfalls wird nicht die gesamte Planung der Internetpr�senz\footnote{vgl. \cite{kyas}, S274} in dieser Arbeit besprochen. Des Weiteren sollten die hier erarbeiteten Konzepte im Einklang mit den Vorgaben der Marketingstrategie stehen. Da hier lediglich ein fiktives Unternehmen betrachtet wird, besteht eine solche Strategie nicht.

Die komplette Realisierung einer echten Unternehmenswebsite w�rde hier noch weitere Planungs-- und Arbeitsschritte erfordern.

Es wird kein Social--Media--Konzept erstellt, da dies den Rahmen der Arbeit �berschreiten w�rde. Allerdings wird die Einbindung von Web 2.0 Komponenten im Rahmen des Redaktionellen Konzepts betrachtet.