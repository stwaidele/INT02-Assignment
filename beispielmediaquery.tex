\section{Detailbetrachtung: Responsive Layout}

\textsc{Matthias Vongerichten}

Die Anpassung an verschiedene Ger�tetypen, vom Mobiltelefon �ber Tablet--PCs und �ltere PCs bis hin zum Modernen PC mit gro�em Monitor wird im Stylesheet mit Hilfe von Media--Queries implementiert.

So nimmt die Website auf kleinen Bildschirmen 90\% der Breite ein. Ab einer Fenstergrbreite von 960 Pixeln w�chst die Breite des genutzten Bereichs jedoch nicht mehr. Hierdurch werden �berlange Zeilen, die nur schwer zu lesen sind vermieden.
Der Lesefluss wird weiterhin erleichtert, in dem ab 800 Pixeln Breite der Zeilenabstand um 20\% erh�ht wird. 

Auf kleinen Bildschrimen nehmen grafische Elemente wie das Logo oder Bilder die ganze zur Verf�gung stehende Breite ein. Hierdurch werden zu gro�e und auch zu kleine Grafiken vermieden und der geringe zur Verf�gung stehende Platz wird optimal ausgenutzt.

Als Beispiel werden in Abbildung~\ref{abb:mq} die �nderungen gezeigt, die bei einer Fensterbreite von 800 Pixeln gemacht werden:

\begin{figure}[h]
\begin{minted}[bgcolor=bg]{css}
@media only screen and (min-width: 800px){
/* Klassischer Desktop */
   p,ol,ul {
      line-height:1.2;
   }
   #sidebar {
      width:25%;
      float:right;
   }
   #content {
      width:70%;
      float:left;
   }
   #content img {
      float:left;
      width:250px;
   }
   .titel {
      width:100%;
   }
}
\end{minted}
\caption{Quellcode: Media Query (CSS)}
\label{abb:mq}
\end{figure}