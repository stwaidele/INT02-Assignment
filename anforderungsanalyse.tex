\section{Anforderungsanalyse}

\subsection{Ziele der Website}

	\begin{itemize}
			\item Gewinnung neuer Kunden
			\item Erh�hun des Bekanntheitsgrades
	\end{itemize}

\subsection{Zielgruppen}

\subsubsection{Zielgruppe "Junge Erwachsene"}

BlaBla

\subsubsection{Zielgruppe "Schulklassen"}

BlaBla

\subsubsection{Zielgruppe "Rentner"}

BlaBla

\subsection{Funktionale Anforderungen}

Folgende Funktionen sollten dem Besucher der Seite angeboten werden:

\begin{itemize}
	\item Vermitteln von Informationen �ber das Unternehmen
	\item Darstellung der Reiseziele und deren Details
	\item Buchung von Reisen
	\item Kontaktaufnahme mit dem Unternehmen
\end{itemize}

\subsection{Nicht-funktionale Anforderungen}

Folgenden nicht-funktionalen bzw.qualitativen Kriterien soll die Website entsprechen:

\begin{itemize}
	\item Abgestimmtes Erscheinungsbild
	\item Kurze Ladezeiten
	\item Pr�gnante Aussagen
	\item Einfache Kontaktaufnahme
	\item Wenig Administrationsaufwand
\end{itemize}

\subsection{Technische Voraussetzungen}

Zum Betreiben der Website wird ein Webspace mit mindestens 50MB Speicherplatz, vorzugsweise unlimitiertem Traffic, FTP-Zugang, und der M�glichkeit der mod_rewrite-Aktivierung ben�tigt. Au�erdem muss der verwendete Webserver PHP interpretieren k�nnen. Eine Datenbank ist nicht erforderlich. F�r das Bearbeiten der Seiteninhalte gen�gt ein Text-Editor.
Zur besseren Auffindbarkeit der Website wird eine Top-Level-Domain empfohlen. Diese kann dem Webspace oft einfach zugebucht werden.