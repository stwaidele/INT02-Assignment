\section{Anforderungsanalyse}

\textsc{Stefan Waidele}

\subsection{Ziele der Website}

Die Unternehmenswebsite f�gt sich in die gro�e Bandbreite der Werbema�nahmen eines Unternehmens ein.
Daher muss diese sich auch nach den Grunds�tzen der Werbung und des Marketings richten.

Die allgemeinen Ziele der Website sind die Gewinnung neuer Kunden sowie die Erh�hung des Bekantheitsgrades des Unternhemens.

F�r diese Arbeit gehen wir von folgendenden Vorgaben\footnote{\cite{kloss}, S. 183ff} aus:

\begin{itemize}
	\item \textbf{Zielgruppe:}\\
	Das Unternehmen betrachtet Alleinreisende und Paare im Alter zwischen 40 und 65 Jahren als die Hauptzielgruppe f�r die angebotenen St�dtereisen. Au�erdem sollen Vereine und Reisegruppen gemischten Alters angesprochen werden. 
	Altersunabh�ngig ist die Zielgruppe haupts�chlich den Sinus-Charakteristiken "`B�rgerliche Mitte"', "`Traditionelle"' oder "`Konservativ--etablierte"' zuzuordnen\footnote{\cite{sinus}}.
	
	\item \textbf{Phase der Buchungsentscheidung:}\\
	Die Website soll dem Kunden zu jedem Zeitpunkt des Entscheidungsvorgangs\footnote{vgl. \cite{freyer}, S105} etwas bieten k�nnen. Es ist jedoch besonders auf die Informations-- und Handlungsphase zu achten, da hier der Einfluss des Unternehmens am direktesten ist.
\end{itemize}

\subsection{Funktionale Anforderungen}

Folgende Funktionen sollten dem Besucher der Seite angeboten werden:

\begin{itemize}
	\item Vermitteln von Informationen �ber das Unternehmen
	\item Darstellung der Reiseziele und deren Details
	\item Buchung von Reisen
	\item Kontaktaufnahme mit dem Unternehmen
\end{itemize}

\subsection{Nicht-funktionale Anforderungen}

Folgenden nicht--funktionalen bzw.qualitativen Kriterien soll die Website entsprechen:

\begin{itemize}
	\item Abgestimmtes Erscheinungsbild
	\item Kurze Ladezeiten
	\item Einfaches Zurechtfinden
	\item Einfache Kontaktaufnahme
	\item Wenig Administrationsaufwand
	\item Mehrsprachigkeit (Deutsch, Englisch)
\end{itemize}

Dar�ber hinaus sind die Dialoggrunds�tze f�r interaktive Systeme gem�� ISO 9241-110 zu beachten\footnote{\cite{bhofmann}}:

\begin{itemize}
	\item Aufgabenangemessenheit
	\item Selbstbeschreibungsf�higkeit
	\item Erwartungskonformit�t
	\item Fehlertoleranz
	\item Steuerbarkeit
	\item Individualisierbarkeit
	\item Lernf�rderlichkeit
\end{itemize}

\subsection{Technische Voraussetzungen}

Zum Betreiben der Website wird ein Webspace mit mindestens 50MB Speicherplatz, vorzugsweise unlimitiertem Traffic, FTP-Zugang, und der M�glichkeit der mod\_rewrite--Aktivierung ben�tigt. Au�erdem muss der verwendete Webserver PHP interpretieren k�nnen. Eine Datenbank ist nicht erforderlich. F�r das Bearbeiten der Seiteninhalte gen�gt ein Text--Editor.
Zur besseren Auffindbarkeit der Website wird eine Top-Level-Domain empfohlen. Diese kann dem Webspace oft einfach zugebucht werden.